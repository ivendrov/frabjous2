\NeedsTeXFormat{LaTeX2e}
\RequirePackage{fix-cm}
% I include ^-these-^ because it allows patches to apply

\documentclass{article}

\usepackage{alltt                                    % I like these
          , multirow
          , booktabs
          , listings
          , graphicx
          , hyperref
          , amsthm}
\usepackage[table]{xcolor}
\usepackage[numbers]{natbib}     % this is a better citation system

\newcommand*{\figref}[1]{{Figure~\ref{#1}}}
\newcommand*{\tblref}[1]{{Table~\ref{#1}}}
\newcommand*{\secref}[1]{{Section~\ref{#1}}}
\newcommand*{\thmref}[1]{{Theorem~\ref{#1}}}
\newcommand*{\corref}[1]{{Corollary~\ref{#1}}}
\newcommand*{\eqnref}[1]{{Equation~\ref{#1}}}

\newcommand*{\code}[1]{\texttt{#1}} %semantic markup = font policy
\newcommand*{\codecomment}[1]{\textsl{\textcolor{red}{{#1}}}}
\newcommand*{\term}[1]{\textit{#1}}
\newcommand*{\tblhead}[1]{\textbf{#1}}

\newtheorem{thm}{Theorem}
\newtheorem{cor}{Corollary}

\begin{document}

\title{My Title
  \\ \large{a \LaTeX\ Sampler for \textsc{cmpt}880}}
\author{Christopher Dutchyn $<$\href{mailto:cjd032@mail.usask.ca}
                                    {\url{cjd032@mail.usask.ca}$>$}}
\maketitle

\begin{abstract}
  An abstract should be written last, comprising four sentences: the thesis statement (appearing near the end of the introduction), a summary statement describing the experimental design (appearing early in the experiment section), a summary statement conveying the experimental results and their interpretation (appearing near the end of the results secion), and a summary of future work enabled by the paper (appearing in the summary section).  I recommend simply copying the appropriate sentences, then polishing their edges to make them flow smoothly as a coherent summary of the paper. Indeed. Yes.
\end{abstract}

This is the introductory section.  It should introduce a \emph{research question}\footnote{See the course textbook\citep{text} if you are not intimately familiar with this term.}, and convey its importance: the \emph{so what} I stress in lecture.  It then focuses on a specific \emph{research problem} related to the research question.  It takes a stand by ending with a \emph{thesis statement} describing a research result which resolves the research question.

\begin{table}[!h]
\caption{Measurements}\label{numtab}
\begin{center}
\rowcolors{1}{}{gray!35}
\begin{tabular}{l r c r@{.}l}
  \toprule
  \multicolumn{2}{c}{\tblhead{Item}}
                 & \tblhead{Description}
                 & \multicolumn{2}{c}{\tblhead{Points}}\\
  \midrule
  point & (\term{pt}) & $/{1}{72.27}$ inches & 1 & 0 \\
  pica  & (\term{pc}) & typewriter measure & 12 & 0 \\
  em    &             & width of a letter \term{m} & \multicolumn{2}{c}{varies with font} \\
  en    &             & width of a letter \term{m} & \multicolumn{2}{c}{varies with font} \\
  ex    &             & height of a letter \term{x} & \multicolumn{2}{c}{varies with font} \\
  inch  & (\term{in}) & English measure & 72 & 27 \\
  \bottomrule
\end{tabular}
\end{center}
\end{table}

Some papers might include a roadmap of the paper, indicating that the next section is background (including previous, but not comparable, experiments), followed by our experiment design section.  Those are followed by a series of results (such as tables like \tblref{numtab}) and interpretations, which are compared and contrasted with others' results.  Last, a summary and future work might be presented.

\section{This is a Section}\label{meaningless}

This is a paragraph.  It introduces a new technical word, \term{frob}, which means to manipulate.  Note that I use semantic (i.e. meaning-based) markup, not syntactic (i.e. format-based) markup.

This is another paragraph, with lots of text.
It continues, with lots of text.
It continues, with lots of text.
It continues, with lots of text.
It continues, with lots of text.
It continues, with lots of text.
It continues, with lots of text.
It continues, with lots of text.
It continues, with lots of text.
It continues, with lots of text.
It continues, with lots of text.
It continues, with lots of text.
It continues, with lots of text.
It continues, with lots of text,
in \secref{meaningless}.

\begin{thm}[Monochromaticity of Horses]\label{horseThm}

All horses are the same color.

\end{thm}
\begin{proof}
  We proceed by induction over the size of the herd.  There are two cases:
  \begin{enumerate}
    \item \emph{$n=1$} A herd of one horse clearly contains horses of precisely one colour.
    \item \emph{induct} Assume the induction hypothesis:
      %
      \[ \textrm{Any herd of } n>1 \textrm{ horses are all the same colour.} \]
      %
      and consider a herd of $n+1$ horses.  Remove one horse, $H_1$, leaving a herd of $n$ horses, \{ $H_2$, \ldots $H_{n+1}$ \}, which by assumption, are all the same colour, that is,
%
\[ H_2 = \ldots = H_n = H_{n+1} \]
%
Then, because $n>1$, we still have a herd of $n>=1$ horses by removing a different horse, $H_{n+1}$.  Now we have
%
\[ H_1 = H_2 = \ldots = H_n \]
%
So, we have that all $n+1$ horses are the same colour:
\[ H_1 = H_2 = \ldots = H_n = H_{n+1} \]
\end{enumerate}

\noindent Hence, by the principle of (weak) mathematical induction, all horses are the same colour.
\end{proof}

\begin{cor}[White Horse Theorem]\label{whiteHorses}
All horses are white.
\end{cor}
\begin{proof}
This is a trivial consequence of the previous theorem, given that George Washington's horse was white.
\end{proof}

There are other surprising consequences of our theorem, including the following, which does not contradict \corref{whiteHorses}.

\begin{cor}[Nonexistence of Horses]
Horses do not exist.
\end{cor}
\begin{proof}
Without loss of generality, consider a horse.  It has fore legs and two more hind.  Since $4+2=6$, that means it has six legs; clearly a \emph{horse of a different colour}~\cite{idiom}.  But this is impossible, by \thmref{horseThm}.  So, no horse can exist.
\end{proof}

As a more truthful example, because it involves mathematical equations, consider the following proof that zero equals one, starting with \eqnref{anEquation}.

\begin{thm}
$1 = 0$.
\end{thm}
\begin{proof}
Assume
\begin{equation}\label{anEquation}
  a = b
\end{equation}
Then,
\begin{eqnarray*}
       a^2 & = & ab \\
      2a^2 & = & a^2 + ab \\
2a^2 - 2ab & = & a^2 + ab - 2ab \\
  2a (a-b) & = & a^2 - ab \\
           & = & a (a-b) \\
        2a & = & a \\
         2 & = & 1 \\
         1 & = & 0
\end{eqnarray*}
\end{proof}

There is a figure, namely \figref{useless}, somewhere.  \LaTeX is notorious for putting them in strange places\footnote{But, \LaTeX will never place it on a page \emph{before} the one containing the text that immediately preceeds it.}.
%
\begin{figure}[!t]
\begin{centering}
\includegraphics[scale=0.3]{happy}
\qquad
\includegraphics[width=0.25\textwidth,height=5ex,angle=45]{happy}
\caption{Figures are Captioned Below}\label{useless}
\end{centering}
\end{figure}
%
There are also tables, as in \tblref{table1}; I make them differently, because book tables are more {\ae}sthetically pleasing\footnote{Also known as \term{beautiful}.}.

\begin{table}[!h]
\caption{Tables are Captioned Above}\label{table1}
\center{
    \begin{tabular}{rl}
      \toprule
      \tblhead{Key} & \tblhead{Value}                          \\
      \midrule
      \term{this} & is a table.                                \\
      \midrule
      \term{what} & contains \textcolor{blue}{aligned} items.  \\
      \midrule
      \term{how}  & using \verb+\includepackage{booktabs}+.    \\
      \midrule
      \multirow{3}*{\term{where}} & here, \\
                                  & there, \\
                                  & everywhere. \\
      \bottomrule
    \end{tabular}
}
\end{table}

This paragraph cites an interesting article~\citep{foo}.
And then we cite another via authors: \citet*{bar} says nothing useful.

\begin{figure}[!h]
\begin{alltt}
          \codecomment{//this is a silly program}
          int main(int argc, char* argv[]) \{
              write(1, argv[0], 7); 
          \}
\end{alltt}
\caption{a Program Directly}\label{prog1}
\end{figure}

Now we're done \ldots
%
\begin{figure}[!b]
\lstinputlisting{prog.c}
\caption{Program Again, as a Listing}\label{prog2}
\end{figure}
%
except for two programs, defining \code{main()} in two ways.  The first, \figref{prog1}, inserts source directly in the \LaTeX file, using the \code{alltt} environment, and the other, \figref{prog2}, uses the \code{listings} package to read the code from an external source file and place it at a page.

\section*{an Unnumbered Section}

There are many details about \LaTeX that we can discover as future work, c.f.~\citet{lshort}.  For example,
\begin{itemize}
  \item special settings for two-sided printing;
  \item struts to adjust individual inter-line spacing;
  \item the \code{vfill} and {hspace} commands to add flexible space;
  \item layout-dependent dimensions, such as \code{baselineskip} and \code{textwidth} to adjust for new sizes;
  \item sophisticated document classes like \code{memo} for production-quality books;
  \item \code{xypic} package for most line-drawing needs
  \item \code{subfloat} for complex multi-figure and multi-caption figures;
  \item other font sizes, like {\Huge{Huge}}, {\LARGE{LARGE}}, {\Large{Large}}, {\large{large}}, {\small{small}}, {\footnotesize{footnotesize}}, and {\tiny{tiny}};
  \item hundreds of other font faces;
  \item drop capitals,
  \item every arrow, operator, and other symbol under the sun,
  \item packages to typeset chess, go, checkers, and music;
  \item headers and footers for placing chapter titles and book titles into the margins;
  \item marginal notes;
  \item change bars;
  \item macros to define new environments; and
  \item colours for backgrounds.
\end{itemize}
For now, the examples found here suffice.

\bibliographystyle{abbrvnat}
\bibliography{test}

\rule{0.5\textwidth}{2pt}
\textit{
\begin{itemize}
  \item November 10, 2012: initial release for \textsc{cmpt}880
\end{itemize}
}
\end{document}
% v-this-v lets me use the space below for cut-and-paste text
\endinput

This is a scratch area where I can save an old version of stuff to
use ...

